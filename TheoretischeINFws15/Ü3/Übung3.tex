

\documentclass[12pt]{article}

\usepackage[german]{babel}
\usepackage[utf8]{inputenc}
\usepackage{lmodern}
\usepackage{amsmath}
\newcommand{\exercise}{Aufgabe}
\usepackage{array}
\usepackage{amstext}
\def \r {\rightarrow}
\newcolumntype{C}{>{$}c<{$}}
\begin{document}
  
\begin{center}
\begin{huge}
\textbf{Übungsblatt 3}\\
\textit{ Sprachen und Grammatiken
}\end{huge} \\

Theoretische Informatik\\
Studiengang Angewandte Informatik\\
Wintersemester 2015/2016\\
Prof. Barbara Staehle, HTWG Konstanz
\end{center}
\section*{Aufgabe 3.1}  
 topic={Grammatiken, Ableitungen und Syntaxbäume für $D_3$} 
 ] 
  
 \subsection*{Teilaufgabe 3.1.1} [ 
 topic = {Eine Grammatik für die Dyck-Sprache $D_3$}, 
 credits = 1 
 ] 
 Aus der Vorlesung ist Ihnen die Dyck-Sprache $D_2$ bekannt, sowie eine Grammatik $G_2$ mit $\mathcal{L}(G_2) = D_2$.  
  
 Geben Sie die Grammatik $G_3$, welche die Sprache $D_3$ (alle korrekt geklammerten Ausdrücke mit den Klammerpaaren (), [], \{ \} ) erzeugt an.   \\\\
 \begin{align*}
 G_3 = \{N,\Sigma,P,S\} = \{\{S\},\{(,),[,],\{,\}\},P,S\}\\
 S \rightarrow \epsilon | SS | (S) | [S] | \{S\}\\
 \mathcal{L(D_3)} = \{\epsilon,[],(),([]),\{\},\{()\},\ldots \}
 \end{align*}
  
  
 \subsection*{Teilaufgabe 3.1.2} [ 
 topic = {Ableitung des Wortes $\{([])()\}[]$ }, 
 credits = 2 
 ] 
  
 Geben Sie eine Linksableitung des Wortes $\{([])()\}[]$ an. \\\\
  $
S \rightarrow SS\\
	\rightarrow \{S\}S\\
	\rightarrow \{SS\}S\\
	\rightarrow \{(S)S\}S\\  
	\rightarrow \{([S])S\}S\\
	\rightarrow \{([])(S)\}S\\
	\rightarrow \{([])()\}S\\
	\rightarrow \{([])()\}[S]\\
		\rightarrow \{([])()\}[]
  $
  
 \subsection*{Teilaufgabe 3.1.3} [ 
 topic = {Syntaxbaum zur Ableitung des Wortes $\{([])()\}[]$ }, 
 credits = 2 
 ] 
  
 Geben Sie für Ihre Linksableitung des Wortes $\{([])()\}[]$ den dazugehörigen Syntaxbaum an. 
  
\textit{  \emph{SIEHE BLATT}
} \section*{Aufgabe 3.2} [ 
 topic = Die Chomsky-Hierarchie, 
 credits = 2 
 ] 
  
 Sei $N = \{S,T,U\}$ das Alphabet der Nonterminale, $\Sigma = \{1,2,3\}$ das Alphabet der Terminale über welchem 8 verschiedene Grammatiken definiert sind. Im Folgenden ist aus jeder dieser Grammatiken eine Regel angegeben.  
  
 Geben Sie für jede der Regeln an, von welchem Chomsky-Typ sie (maximal) ist. Wenn also eine Regel vom Typ 0, 1 und 2 ist, dann ist die Lösung "`Typ 2"'. 
  
 Begründen Sie Ihre Entscheidung.  \\
  $--------------------------------$
\textbf{\textit{ALLGEMEIN} Typen: Chaomsky
}
 \textit{Bemerkungen:} immer auf 2 und 3 zuerst nachchecken

\begin{enumerate}
\item ES GILT $ N = \{S,T\} , \Sigma =\{a,b,c\}$ \\ 
allgemeine Form $l \to r$ mit $l \in (N \cup \Sigma)^+$ und $r \in (N \cup \Sigma)^* $ das Sternchen heißt mit leerem Wort $\epsilon$
\item   Typ 0 : keine Einschränkungen \\
$aSb \to Ta$

\item   Typ 1 : Kontextsensetiv \\
Länge r muss größer gleich die Länge von l\\
$|r| \geq |l|$\\
Z.B : $aSb \to aTcb$\\
Ausnahme : $S \to \epsilon$ erlaubt falls S nicht auf der rechten Seite einer Regel vorkommt.
\item   Typ 2: kontextfrei, vom Typ 1 jedoch gilt dass $l \in N$ rechte Seite egal \\
    $S \rightarrow aSb$\\
    Ausnahme von Typ 1 gilt hier auch
\item Typ 3: regulär \\ linke Seite : ein Nonterminal Symbol \\ rechte Seite : leeres Wort oder einzelnes Terminalsymbol oder einzelnes Terminalsymbol folgt von einem NonTerminal.\\
$S \to aT$\\
$S \to a$\\
$S \to \epsilon$

     Ausnahme gilt
     
\end{enumerate}
$--------------------------------------$\\ \\

     
 \begin{enumerate} 
 \item $r: T \rightarrow 1$ \\ regulär Typ 3 wegen $A \rightarrow a$
 \item $s: T \rightarrow 12$ \\ typ 2 
 \item $t: ST \rightarrow 12$\\  Typ 1
 \item $u: ST \rightarrow 1$ \\ Typ 0 
 \item $v: R2S \rightarrow 23T$ \\  Typ 1
 \item $w: R \rightarrow S$\\ Typ 2
 \item $x: R \rightarrow 1S$ \\ Typ 3
 \item $y: 2RST \rightarrow R3R$ \\  Typ 0 (die Längen vergleichen!)
 \end{enumerate} 
  
 \textbf{Zusatzfrage (ohne Punkte)} Handelt es sich bei der Regel $z: 1 \rightarrow 1S$ Ihrer Meinung nach um eine korrekte Typ-0 Regel? Analysieren Sie einerseits die Definition, aber bedenken Sie auch, was Sie generell über Grammatiken und Regeln wissen.  
  \\\\
   Nonterminal Symbole sollten auf der linken Seite sein damit diese eine korrekte Typ 0 Grammatik ist 
  
  
  
 \section*{Aufgabe 3.3} [ 
 topic = Zahlensprachen 
 ] 
  
 \subsection*{Teilaufgabe 3.3.1} [ 
 topic = Die Sprache der natürlichen Zahlen, 
 credits = 2 
 ] 
  
 $L_N \subseteq \{0,1,\ldots,9\}^*$ mit $L_N = \{0, 1, \ldots 9, 10, \ldots, 5906, \ldots, \}$ sei die Sprache der natürlichen Zahlen. 
 \begin{enumerate} 
 \item Geben Sie eine Grammatik an, welche $L_N$ erzeugt.  
 \\
 $G_N = \{\{S,N\},\{0,...9\},P,S\}$ mit P =\\
 
 $S \rightarrow 0|...|9|1N|...|9N$
 
 $N \rightarrow 0|..|9|1S|......|9S$
 \item Welchen Chomsky-Typ hat Ihre Grammatik? \\
 Typ 3
 \item Können Sie Ihre Grammatik so umformen, dass sie regulär ist? \\
 Sie ist von Typ 3 also regulär, muss nicht umgeformt werden
 \end{enumerate}  
  
  
  
  
 \subsection*{Teilaufgabe 3.3.2} [ 
 topic = Die OTTO-Zahlen, 
 credits = 3 
 ] 
  
 $L_O \subseteq L_N \subseteq \{0,1,\ldots,9\}^*$ mit $L_O = \{0, 1, \ldots 9, 11, 22 \ldots , 99, 101, 111, 121, \ldots , 573375, \ldots \}$, sei die Sprache der OTTO-Zahlen, also der natürlichen Zahlen, die von vorne und hinten gelesen gleich sind. 
 \begin{enumerate} 
 \item Geben Sie eine Grammatik an, welche $L_O$ erzeugt.\\ 
 $G_N = \{\{S,A,B,C,D,E,F,G,H,I,J\},\{0,...9\},P,S\}$ mit P =\\
 $S \rightarrow 0|...|9|0S|1S|...|9S$\\
 \item Welchen Chomsky-Typ hat Ihre Grammatik? 
 Typ 3
 \item Können Sie Ihre Grammatik so umformen, dass sie regulär ist? 
 ist auch Typ 3
 \end{enumerate}  
  
  
 \end{document} 
  
 