

\documentclass[12pt]{article}

\usepackage[german]{babel}
\usepackage[utf8]{inputenc}
\usepackage{lmodern}
\usepackage{amsmath}
\usepackage{amssymb}
\usepackage{array}
\usepackage{tikz}
\usepackage{colortbl}
\usepackage{xcolor}
\usetikzlibrary{arrows,automata}
\usepackage{pgf}
\usepackage{listings}
\usepackage{amstext}
\def \nat {\mathbf{N}}
\newcolumntype{C}{>{$}c<{$}}
\newcolumntype{R}{>{$}r<{$}}
\usepackage{hyperref}
\usepackage{stmaryrd}
\begin{document}
  
\begin{center}
\begin{huge}
\textbf{Übungsblatt 8}\\
\textit{Programme und Rekursion}
\end{huge} \\

Theoretische Informatik\\
Studiengang Angewandte Informatik\\
Wintersemester 2015/2016\\
Prof. Barbara Staehle, HTWG Konstanz
\end{center}

\section*{Aufgabe 8.1}

\textit{[ 
 topic = Berechnung der Subtraktion 
 ] }
  
 Sei die "natürliche Subtraktion" definiert als $-: \nat_0 \times \nat_0 \rightarrow \nat_0, (x-y) =  
 \left\{ 
 \begin{array}{ll} 
 x-y & \text{ falls } x \geq y, \\ 
 0 & \text{ falls } x < y. 
 \end{array} 
 \right . $   
  
 Es gilt also z.B. $4-3 = 1, 7-5 = 2, 2-3 = 0, 1-9 = 0$. 
  
 \subsection*{Teilaufgabe 8.1.1} 

 Erstellen Sie das Loop-Programm sub.loop mit $x_0:=\text{sub}(x_1,x_2)$ welches als Ergebnis $x_1 - x_2$ (wie oben definiert) zurückliefert. \\
  \emph{Lösung}\\
\begin{lstlisting}[frame=single] 
x0 := x1;
LOOP x2 DO 
 x0 := pred(x0)
END
\end{lstlisting}



  
  
 \subsection*{Teilaufgabe 8.1.2} 

  
 Berechnen Sie für den Vektor $\nu = (0,5,3)$ und Ihr Programm $P: x_0:=\text{sub}(x_1,x_2)$ das Ergebnis von $\delta(\nu,P)$. Geben Sie insbesondere alle ca. 10 Zwischenschritte Ihrer Berechnung an.  
  \\ 
\textbf{  Lösung:
}  \\
$\delta(v,P)$\\
$=\delta((0,5,3),P)$\\
$=\delta(\alpha((0,5,3),x_0 := x_1),\ LOOP \ x_2 \ DO \ pred(x_0) \ END)$\\
$=\delta(\alpha((0,5,3),x_0 := x_1),\ LOOP \ x_2 \ DO \ x_0 := \ pred(x_0) \ END)$\\
$=\delta((5,5,3),\ LOOP \ x_2 \ DO 
\ pred(x_0) \ END)$\\
  $=\delta((5,5,3),x_0 := pred(x_0);x_0 := pred(x_0);x_0 := pred(x_0))$\\
    $=\delta(\alpha((5,5,3),x_0 := pred(x_0);x_0 := pred(x_0);x_0 := pred(x_0)))$\\
     $=\delta((4,5,3),x_0 := pred(x_0);x_0 := pred(x_0))$\\
       $=\delta(\alpha((4,5,3),x_0 := pred(x_0);x_0 := pred(x_0)))$\\
      $=\delta((3,5,3),x_0 := pred(x_0))$\\
      $=(2,5,3)$
 \subsection*{Teilaufgabe 8.1.3} 

  
 Geben Sie nun mit Hilfe der vereinfachten Notation, die Sie in der Vorlesung kennengelernt haben, die Zustandsänderungen des Eingabevektors $\nu$ bis das Programm beendet ist in den beiden folgenden Fällen an:  
\emph{  Tupel untereinander schreiben ohne $=$!
} \begin{enumerate} 
 \item $\nu = (0,3,5)$ \\
\textbf{ \Lösung
}\\
$$ (0,3,5)\overset{initial  \ x_0 = x_1}{=}(3,3,5)\overset{x2=5}{=}(2,3,5)\overset{x2=4}{=}(1,3,5)\overset{x2=3}{=}(0,3,5)\overset{x2=2}{=}(0,3,5)\overset{x2=1}{=}(0,3,5)$$ 


\item $\nu = (0,5,0)$ \\
\textbf{ \Lösung
}\\
$x_2$ ist $0 \rightarrow$ LOOP läuft $0$ mal\\
$$ (0,5,0)\overset{initial \ x_0 = x_1}{=}(5,5,0)$$ 
 \end{enumerate} 
  
 Hinweis: Geben Sie alle Zustandsänderungen an, auch wenn sich der Nachfolgezustand nicht vom Vorgängerzustand unterscheidet. 
  
  
  
 \subsection*{Teilaufgabe 8.1.4} 
\textit{[ 
 topic = Subtraktion als primitiv-rekursive Funktion, 
 credits = 1 
 ] }
  
 Die Subtraktion ist auch eine primitiv rekursive Funktion. Wenn Sie sub($x_1,x_2$) als primitiv-rekursive Funktion schreiben möchten, welche Sachverhalte könnten Sie sich zu Nutze machen, bzw. welche Ideen erscheinen Ihnen nützlich? 
  
 Hinweis: Es ist \textbf{NICHT} verlangt, dass Sie die Subtraktion tatsächlich als primitiv-rekursive Funktion angeben. \\  \textbf{Lösung}:\\
 $sub(x_1,x_2)=sub(x_1,x_2-1)-1$
  

 \section*{Aufgabe 8.2} 
\textit{ [
 topic = Berechnung der Fakultät 
 ] }
  
 Bevor Sie mit der Bearbeitung der Aufgaben beginnen, machen Sie sich anhand der Beispiele $4!, 5!, 7!$ klar, wie viele Multiplikationen Sie für die Berechnung von $n!$ jeweils durchführen müssen.  
  
  
 \subsection*{Teilaufgabe 8.2.1} 
\textit{[ 
 topic = factorial.loop, 
 credits = 3 
 ] }
  
 Erstellen Sie das Loop-Programm factorial.loop mit $x0 := \text{factorial}(x1)$, welches als Ergebnis $x_1!$ zurückliefert. 
  
 Hinweise: 
 \begin{itemize} 
 \item Es ist eventuell von Vorteil, während des Programmablaufs die Hilfsvariable $x_2$ zu verwenden. 
 \item Achten Sie darauf, dass $1! = 0! = 1$ korrekt berechnet wird.  
 \end{itemize} 
\emph{  Lösung:}  
  \begin{lstlisting}[frame=single] 
x0 := 1;
x2 = x1-1;
LOOP x2 DO 
 x0 := mult(x0,x1)
 x1 := pred(x1)
END
\end{lstlisting}
  
 \subsection*{Teilaufgabe 8.2.2} 
\textit{[ 
 topic = Arbeitsweise von factorial.loop, 
 credits = 1 
 ] }
  
 Vollziehen Sie anhand der Veränderungen im Eingabevektor $\nu =(0,3,0)$ die Arbeitsweise Ihres Programms für die Berechnung von 3! nach.  \\
 \textbf{ \Lösung
}\\
$$(0,3,0) 
 (1,3,3)  (3,3,3)  (3,3,2)  (3,3,2) (6,3,2)  (6,3,1)  (6,3,1)  (6,3,0)$$  
  
  
  
  \subsection*{Teilaufgabe 8.2.3} 
\textit{[ 
 topic = factorial.while, 
 credits = 2 
 ] }
  
 Erstellen Sie das While-Programm factorial.while mit $x_0 := \text{factorial}(x_1)$, welches als Ergebnis $x_1!$ zurückliefert. Erstellen Sie Ihre Programm als echtes While-Programm ohne die Verwendung der Loop-Schleife. 
  
 Hinweis: Achten Sie darauf, dass $1! = 0! = 1$ korrekt berechnet wird. \\ 
  \emph{  Lösung:}  
  \begin{lstlisting}[frame=single] 
x0 := 1;
WHILE x1  DO 
 x0 := mult(x0,x1)
 x1 := pred(x1)
END
\end{lstlisting}
  
  
  
  \subsection*{Teilaufgabe 8.2.4} 
\textit{[ 
 topic = Arbeitsweise von factorial.while, 
 credits = 1 
 ]} 
  
 Vollziehen Sie anhand der Veränderungen im Eingabevektor $\nu =(0,3)$ die Arbeitsweise Ihres Programms für die Berechnung von 3! nach.\\
 \textbf{ \Lösung
}\\
$$(0,3,0) 
 (1,3,3)  (3,3,3)  (3,3,2)  (3,3,2) (6,3,2)  (6,3,1)  (6,3,1)  $$    
  
  
 \subsection*{Teilaufgabe 8.2.5} 
\textit{[ 
 topic = Fakultät als primitiv-rekursive Funktion, 
 credits = 1 
 ] }
  
 Die Fakultät ist eine primitiv rekursive Funktion. Wenn Sie factorial($x_1,x_2$) als primitiv-rekursive Funktion schreiben möchten, welche Sachverhalte könnten Sie sich zu Nutze machen, bzw. welche Ideen erscheinen Ihnen nützlich? 
  
 Hinweis: Es ist \textbf{NICHT} verlangt, dass Sie die Fakultät tatsächlich als primitiv-rekursive Funktion angeben. 
  
  
  
  
  
 \end{document} 
  
 