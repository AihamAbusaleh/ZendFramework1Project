
\documentclass[12pt]{article}

\usepackage[german]{babel}
\usepackage[utf8]{inputenc}
\usepackage{lmodern}
\usepackage{amsmath}
\usepackage{amssymb}
\usepackage{array}
\usepackage{tikz}
\usepackage{colortbl}
\usepackage{xcolor}
\usetikzlibrary{arrows,automata}
\usepackage{pgf}
\usepackage{listings}
\usepackage{amstext}
\def \nat {\mathbf{N}}
\newcolumntype{C}{>{$}c<{$}}
\newcolumntype{R}{>{$}r<{$}}
\usepackage{hyperref}
\usepackage{stmaryrd}
\begin{document}
  
\begin{center}
\begin{huge}
\textbf{Übungsblatt 9}\\
\textit{Turing Maschinen  }
\end{huge} \\

Theoretische Informatik\\
Studiengang Angewandte Informatik\\
Wintersemester 2015/2016\\
Prof. Barbara Staehle, HTWG Konstanz
\end{center}
 

\section*{Aufgabe 9.1}
\subsection*{Aufgabe 9.1.1}


  
 \begin{figure}[h] 
   $T_2 = (\{s_0,s_1,s_2,s_3\},\{1\},\{1,\Box\},\delta,s_0,\Box.s_3)$ \\ mit $\delta =$ \\

 \centering 
  
 \begin{tikzpicture}[shorten >=3pt,node distance=2.0cm,auto, font=\footnotesize] 
  
 \tikzstyle{every state}=[minimum size = 0cm,inner sep=3pt] 
 \node[state,initial]   (A) {$s_0$}; 
 \node[state] 			(B) [ right of = A]{$s_1$};
 \node[state] 			(C) [ below of = B]{$s_2$};
 \node[state,accepting] (D) [ left of = C]{$s_3$};


 
 \path[->] (A) edge  node {$\Box;\Box,\leftarrow$} (B)
  		   	   edge [loop above] node {1;1,$\rightarrow$} (A) 
  		   (B) edge   node {1;$\Box,\leftarrow$} (C)
  		   	   edge   node[anchor=above,left] {$\Box;\Box,\circlearrowleft$} (D) 
  		   (C) edge   node {1;$\Box,\leftarrow$} (D);

 \node [below left of = A] {$T_2$}; 
   
  
 \end{tikzpicture} 
 \end{figure} 

 $(s_0,1)=(s_0,1,\rightarrow)$ \ \text{Einsen durchlaufen}\\
 $(s_0,\Box)=(s_1,\Box,\leftarrow)$\ \text{Endzeichen lesen und zurück} \\
 $(s_1,\Box)=(s_3,\Box,\circlearrowleft)$\ \text{für f(0)} \\
 $(s_1,1)=(s_2,\Box,\leftarrow)$\ \text{lösche letzte 1}\\
$(s_2,1)=(s_3,\Box,\leftarrow)$\ \text{lösche vorletzte 1 und gehe in Endzustand} \\

\begin{tabular}{|C||C|C|}
\delta & 1 & \Box \\ \hline
s_0 & (s_0,1,\rightarrow) & (s_1,\Box,\leftarrow)\\ 
s_1 & (s_2,\Box,\leftarrow)&(s_3,\Box,\circlearrowleft)\\
s_2 & (s_3,\Box,\leftarrow)&- \\
s_3 & - & -
\end{tabular}

 \subsection*{Aufgabe 9.1.2}
 \begin{enumerate}
 \item $w=\epsilon$\\
$\vdash$  $(\Box,s_0,\Box)$ Start \\
$\vdash$   $(\Box \Box,s_1,\Box)$ Leerzeichen gefunden, laufe links und wechsele Zustand\\
$\vdash$  $(\Box,s_3,\Box)$ Endkonfiguration
 \item $w=1$\\
  $(\Box,s_0,1)$ Start \\
 $\vdash$   $(\Box 1,s_0,\Box)$ 1 gefunden laufe nach rechts \\
 $\vdash$     $(\Box 1,s_1,\Box)$ Leerzeichen gefunden laufe nach links \\
  $\vdash$     $(\Box \Box,s_3,\Box)$ Leerzeichen gefunden schreibe $\Box$ und gehe in Endzustand  \\
\item $w=11$\\
 $(\Box,s_0,11)$ Start \\
 $\vdash$   $(\Box 1,s_0,1)$ 1 gefunden laufe nach rechts \\
  $\vdash$   $(\Box 11,s_0,\Box)$ 1 gefunden laufe nach rechts \\
 $\vdash$     $(\Box 11,s_1,\Box)$ Leerzeichen gefunden laufe nach links \\
  $\vdash$     $(\Box \Box 1,s_2,\Box)$ Leerzeichen gefunden schreibe $\Box$    \\
    $\vdash$     $(\Box \Box \Box,s_3,\Box)$ Leerzeichen gefunden schreibe $\Box$ und gehe in Endzustand  \\
    
    \item $w=111$\\
 $(\Box,s_0,111)$ Start \\
 $\vdash$   $(\Box 1,s_0,11)$ 1 gefunden laufe nach rechts \\
  $\vdash$   $(\Box 11,s_0,1)$ 1 gefunden laufe nach rechts \\
    $\vdash$   $(\Box 111,s_0,1)$ 1 gefunden laufe nach rechts \\
 $\vdash$     $(\Box \Box11,s_1,\Box)$ Leerzeichen gefunden laufe nach links \\
  $\vdash$     $(\Box \Box \Box,s_2,1)$ 1 gefunden schreibe $\Box$ und gehe links   \\
    $\vdash$     $(\Box \Box \Box \Box ,s_3,1)$ 1 gefunden schreibe $\Box$ und gehe in Endzustand  \\
 \end{enumerate}
  \subsection*{Aufgabe 9.1.3}
  Die Idee bei $T_m$ wäre, dass man so viele Einsen löscht, wie das $m$ groß ist, also bei $T_2$ wurden 2 Einsen von Hinten gelöscht


  \section*{Aufgabe 9.2}
  \subsection*{Aufgabe 9.2.1}
  \begin{tabular}{|C||C|C|}
  Zahl& n & n+1 \\ \hline 
  0& 000&001 \\
  1& 001&010\\
  2&010&011\\
  3&011&100\\
  4&100&101\\
  5&101&110\\
  6&110&111\\
  7&111&1000\\
  8&1000&1001\\
  9&1001&1010\\
  10&1010&1011\\
  11&1011&1100\\
  12&1100&1101\\
  13&1101&1110\\
  14&1110&1111\\
  15&1111&10000
  \end{tabular}
  \subsection*{Aufgabe 9.2.2}
  Die Zahlen 0 und 1 werden invertiert, dann wird die erste Zahl ganz links alleine invertiert.
  \subsection*{Aufgabe 9.2.3}
  
  
 \begin{figure}[h] 
   $T_{n+1} = (\{s_0,s_1,s_2\},\{0,1\},\{0,1,\Box\},\delta,s_0,\Box,s_2)$ \\ mit $\delta =$ \\

 \centering 
  
 \begin{tikzpicture}[shorten >=3pt,node distance=2.0cm,auto, font=\footnotesize] 
  
 \tikzstyle{every state}=[minimum size = 0cm,inner sep=3pt] 
 \node[state,initial]   (A) {$s_0$}; 
 \node[state] 			(B) [ right of = A]{$s_1$};
 \node[state,accepting] (C) [ below of = B]{$s_2$};


 
 \path[->] (A) edge  node {$0;1,\leftarrow$} (B)
  		   	   edge [loop above] node {1;0,$\leftarrow$} (A) 
  		   	   (A) edge  node[anchor=left,left] {$\Box;1,\circlearrowleft$} (C)
  		   (B) edge [loop above]  node[anchor=right,right] {0;0,$\leftarrow$}
  		   node {1;1,$\leftarrow$} (B)
  		   	   edge   node[anchor=above,right] {$\Box;\Box,\rightarrow$} (C) 
  		   (C) edge  [loop below] node[anchor=above,left] {$\Box;\Box,\leftarrow$} node[anchor=above,right] {1;1,$\leftarrow$} node[anchor=below,below] {0;0,$\leftarrow$} (C);

 \node [below left of = A] {$T_{n+1}$}; 
   
  
 \end{tikzpicture} 
 \end{figure} 

 $(s_0,0)=(s_1,1,\leftarrow)$ \\
 $(s_0,1)=(s_0,0,\leftarrow)$ \\
  $(s_0,\Box)=(s_2,1,\circlearrowleft)$ \\
 $(s_1,0)=(s_1,0,\leftarrow)$ \\
 $(s_1,1)=(s_1,1,\leftarrow)$ \\
  $(s_1,\Box)=(s_2,\Box,\rightarrow)$ \\
$(s_2,0)=(s_2,0,\circlearrowleft)$\\
$(s_2,1)=(s_2,1,\circlearrowleft)$\\
$(s_2,\Box)=(s_2,\Box,\circlearrowleft)$\\

\begin{tabular}{|C||C|C|C|}
\delta & 0 & 1 & \Box \\ \hline
s_0 & (s_1,1,\leftarrow) & (s_0,0,\leftarrow&(s_2,1,\circlearrowleft)\\ 
s_1 & (s_1,0,\leftarrow&(s_1,1,\leftarrow)&(s_2,\Box,\rightarrow)\\
s_2 & (s_2,0,\circlearrowleft)&(s_2,1,\circlearrowleft)&(s_2,\Box,\circlearrowleft) \\
\end{tabular}
 \subsection*{Aufgabe 9.2.4}
 \begin{enumerate}
 \item $w=\epsilon$\\
 \item $w=0$\\
 $(\Box,s_0,000)$ Start \\
$\vdash$   $(\Box ,s_2,001)$  
 \item $w=1$\\
 $(\Box,s_0,001)$ Start \\
$\vdash$   $(\Box ,s_2,011)$  
 \item $w=111$\\
 $(\Box,s_0,111)$ Start \\
$\vdash$   $(\Box ,s_2,1000)$  
 \end{enumerate}


 \end{document} 
  
 