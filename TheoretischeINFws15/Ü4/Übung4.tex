

\documentclass[12pt]{article}

\usepackage[german]{babel}
\usepackage[utf8]{inputenc}
\usepackage{lmodern}
\usepackage{amsmath}
\newcommand{\exercise}{Aufgabe}
\usepackage{array}
\usepackage{amstext}
\def \nat {\mathbf{N}}
\newcolumntype{C}{>{$}c<{$}}
\usepackage{hyperref}
\begin{document}
  
\begin{center}
\begin{huge}
\textbf{Übungsblatt 4}\\
\textit{Sprachen, Grammatiken und
reguläre Ausdrücke}
\end{huge} \\

Theoretische Informatik\\
Studiengang Angewandte Informatik\\
Wintersemester 2015/2016\\
Prof. Barbara Staehle, HTWG Konstanz
\end{center}
 
\emph{\textbf{\textit{DIESE ÜBUNG WURDE  AUF RICHTIGKEIT NICHT GEPRÜFT}}}
 \section*{Aufgabe 4.1}
\emph{  Pumping-Lemma 
} 
  
\subsection*{Teilaufgabe 4.1.1} 
\emph{ Pumping-Lemma für reguläre Sprachen, 
 credits = 2, 
 label = pumpingReg }
 \\\\
In der Vorlesung wurde mit Hilfe des Pumping-Lemmas bewiesen, dass die Sprache
 $L_{C_2} = \{a^n b^n | n \in N \}$ nicht regulär ist.   
  
 Können Sie diesen Beweis nutzen, um zu entscheiden, ob die Sprache  
 \[ 
 L_1 = \{\omega \in \{a,b\}^* | |\omega|_a = |\omega|_b \} = \{ab, ba, \ldots, babbaaabab, \ldots \} 
 \]  
 (also die Sprache, welche alle Wörter mit gleich vielen $a$s und $b$s enthält) regulär ist?  
  
 Egal ob Sie diesen Beweis nutzen oder nicht - entscheiden Sie, ob $L_1$ regulär ist oder nicht und begründen Sie dies.   
  
\textbf{Lösung :  }  

angenommen ist sie regulär, betrachten wir zunächst das in $L_1$ enthaltene Wort $w = b^n a^m$ mit $m = n$.
wir zerlegen das obige Wort in : \\
\textit{Fall 1 : } 
$$ u = b^i \quad v = b^j \quad w=b^ka \quad \text{Mit} \quad i+j+k = m$$\\
Das Aufpumpen von v führt zum Widerspruch 
$uv^2w = b^i(b^j)^2b^ka =b^ib^{2j}b^ka$ denn i+2j+k $\not$= m \\
\\\\
\textit{Fall 2 : } 
$$ u = b^i \quad v = b^ja \quad w=\epsilon \quad \text{Mit} \quad i+j = m$$\\
Das Aufpumpen von v führt zum Widerspruch 
$uv^2w = b^i(b^ja)^2 =b^ib^{2j}a$ denn i+2j $\not$= m \\
Das heißt dass die Sprache nicht regulär ist.




 
 \subsection*{Teilaufgabe 4.1.2} 
 \emph{topic = Pumping-Lemma für kontextfreie Sprachen, 
 credits = 2, 
 label = pumpingKF 
 ] }
 In der Vorlesung wurde mit Hilfe des Pumping-Lemmas bewiesen, dass die Sprache $L_{C_1} = \{a^n b^n c^n  | n \in \nat \}$ nicht kontextfrei ist.   
  
 Können Sie diesen Beweis nutzen, um zu entscheiden, ob die Sprache  
 \[ 
 L_2 = \{a^i b^j c^k | i \leq j \leq k, i,j,k \in \nat \} = \{abbcccc, cba, \ldots, bcccbacabacb, \ldots \} 
 \]  
 (also die Sprache, welche alle Wörter mit höchstens so vielen $a$s wie $b$s und höchstens so vielen $b$s wie $c$s enthält) kontextfrei ist?  
  
 Egal ob Sie diesen Beweis nutzen oder nicht - entscheiden Sie, ob $L_2$ kontextfrei ist oder nicht und begründen Sie dies.   
  
  \textbf{Lösung :  }  

angenommen ist sie kontextfrei, betrachten wir zunächst das in $L_2$ enthaltene Wort $w = abbcccc$.
wir zerlegen das obige Wort in : \\
\textit{Fall : } 
$$ u = ab^l \quad v = b^m \quad w=b^n \quad x=b^pcc \quad y=cc \quad \text{Mit} \quad l+m+n+p = j$$\\
Das Aufpumpen von v führt zum Widerspruch 
$uv^0wx^0y = ab^lb^pcc $ denn l+p $\not$= j \\Ausßerdem ist in diesem Wort $|j|=|k|$, was die Bedingung verletzt.
\\
Das heißt dass die Sprache nicht regulär ist.

  
  
  
\section*{Aufgabe 4.2} 
\emph{ topic = Eine CNF für die Dyck-Sprache, 
 credits = 3 } 
  
 Aus der Vorlesung ist Ihnen die folgende Grammatik zur Erzeugung der Dyck-Sprache $D_2$ bekannt: 
 \[ 
 S\rightarrow \varepsilon | SS | [S] | (S)  
 \]  
  
 \begin{enumerate} 
 \item Modifizieren Sie die Grammatik so, dass sich das leere Wort nicht mehr ableiten lässt. 
\textbf{ Lösung:
}  \[ 
 S\rightarrow  SS | [S] | (S)  
 \]
 \item Übersetzen Sie die modifizierte Grammatik in Chomsky-Normalform. 
 \textbf{ Lösung:}
 \begin{enumerate}
 \item Elimination 
 $$ S\rightarrow  SS | S | S | [S] | (S) | [] | () $$
  \item Kettenregel
 $$ S\rightarrow  SS | [S] | (S) | [] | () | [S] | (S) | [] | () | [S] | (S) | [] | () $$
 \item Separation
  $$ S\rightarrow  SS | N_[SN_] | N_(SN_) | [] | () $$
  $$ N_[ \rightarrow [ $$
$$ N_] \rightarrow ] $$
$$ N_( \rightarrow ( $$
$$ N_) \rightarrow ) $$  
  \item  Elimination von mehrelementigen Nonterminalkette\\\\
  nix zu tun 
  
 \end{enumerate}

 \item Generieren Sie mit Hilfe der Grammatik in Chomsky-Normalform eine Ableitung und den dazugehörigen Ableitungsbaum für das Wort $([])[]$  
 \textbf{ Lösung:}
 siehe Blatt 
 \end{enumerate} 
  
  
  
 \section*{Aufgabe 4.3}  
\emph{ topic={Reguläre Ausdrücke} }
 Um diese Aufgabe lösen zu können, verwenden Sie die in der Vorlesung verwendete Datei \\
  $thInf_2-formaleSprachen_beispiel_addresslist.txt$ welche in Moodle zur Verfügung steht.
  Außerdem benötigen Sie eine Linux Unix Shell, Cygwin unter Windows oder eine andere Windows-Grep-Lösung, oder ein Online-Tool wie z.B. \href{http://www.online-utility.org/text/grep.jsp}{Grep Online}. 
  
  
 \subsection*{Teilaufgabe 4.3.1} 
\emph{ topic = {Städte}, 
 credits = 3 
 ]} 
 Finden Sie alle Zeilen, welche die Städte der Tourismus-Büros enthält.  
  
 Beispiel: "`Montgomery, AL 36103"' 
  
  
  
  
 \subsection*{Teilaufgabe 4.3.2} 
\emph{ topic = {Staaten und Büros}, 
 credits = 4 
 ] }
 Finden Sie alle Zeilen, welche den Staat des Tourismus-Büros sowie dessen Namen enthält.  
  
 Beispiel: "`Alabama 	Alabama Bureau of Tourism \& Travel"' 
  
\textbf{ Lösung:}  

Montgomery, AL 36103

Juneau, AK 99811-0801

Phoenix, AZ 85007

Little Rock, AR 72201

Sacramento, CA 95812

Denver, CO 80202

Hartford, CT 06103

Dover, DE 19903

Tallahassee, FL 32302

Honolulu, HI 96804

Boise, ID 83720-0093

Indianapolis, IN 46204

Baton Rouge, LA 70804-9291

Augusta, ME 04333-0059

Baltimore, MD 21202

Boston, MA 02116

Jefferson City, MO 65102

Lincoln, NE 68509-8907

Carson City, NV 89701

Concord, NH 03302

Santa Fe, NM 87501

Albany, NY 12220-0603

Bismark, ND 58505-0825

  Pierre, SD 57501-5070

  Richmond,  VA 23219

  Charleston, WV 25305

  Madison, WI 53708-8690

  Cheyenne, WY 82002
  \section*{Aufgabe 4.4} 
\emph{[ 
 topic = Typ-1- und Typ-0-Sprachen, 
 credits = 4 
 ] }
  
 Geben Sie (nicht in der Vorlesung erwähnte) Grammatiken $G_1$ und $G_0$ an, welche zu den Chomsky-Typen 1 bzw. 0 gehören.  
  
 Verwenden Sie das Terminalalphabet $\Sigma = \{a,b\}$ und begründen Sie (bei Bedarf auch anhand der Sprachen $L_1 = \mathcal{L}(G_1)$ bzw. $L_0 = \mathcal{L}(G_0)$), weshalb Ihre Grammatiken nicht auch von einem höheren Chomsky-Typ sind.  
  
\textbf{  Lösung:
}  
$G_0=(\{S,B\},\{a,b\},P,S)$ mit P=

$S \rightarrow \epsilon |aSb$

$aS \rightarrow aB$

$B \rightarrow bB$

$B \rightarrow b$ 

Die kann nicht vom Typ 1 sein, wegen der Regel  $S \rightarrow aSb$ da $S \rightarrow \epsilon$ existiert kann S nicht auf der rechten Seite einer anderen Grammatik stehen.Also Typ 0\\\\
$G_1=(\{S,B\},\{a,b\},P,S)$ mit P=

$S \rightarrow \epsilon $

$aS \rightarrow aB$

$B \rightarrow bB$

$B \rightarrow b$ 


Die ist vom Typ 1, nicht vom Typ 2 wegen der Grammatik $aS \rightarrow aB$

 \end{document} 
  
 