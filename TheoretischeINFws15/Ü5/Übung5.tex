

\documentclass[12pt]{article}

\usepackage[german]{babel}
\usepackage[utf8]{inputenc}
\usepackage{lmodern}
\usepackage{amsmath}
\usepackage{amssymb}
\usepackage{array}
\usepackage{tikz}
\usetikzlibrary{arrows,automata}
\usepackage{pgf}
\usepackage{amstext}
\def \nat {\mathbf{N}}
\newcolumntype{C}{>{$}c<{$}}
\usepackage{hyperref}
\begin{document}
  
\begin{center}
\begin{huge}
\textbf{Übungsblatt 5}\\
\textit{DEAs und NEAs}
\end{huge} \\

Theoretische Informatik\\
Studiengang Angewandte Informatik\\
Wintersemester 2015/2016\\
Prof. Barbara Staehle, HTWG Konstanz
\end{center}

\section*{Aufgabe 5.1}
\textbf{[ 
 topic = Erweiterungen von $A_1$ 
 ] }
 \begin{quote}
  \textit{Allgemein bezüglich Überführung von Automaten! \\
   DEA, das mit dem \textbf{Papierkorbsding}, und jeder Knoten hat \textbf{höchstens} einen Übergang mit der eine Beschriftung $(q,a,q')$,von jedem Knoten müssen also \textbf{alle} Beschriftungen (a,b,c...) ausgehen, anderenfalls spricht man von NEA.... }
  
 \end{quote}
 
 Wie betrachten im folgenden die Automaten $A_2,A_3,A_4$, welche Abänderungen des aus der Vorlesung bekannten Automaten $A_1$ sind. In Abbildung~/ref{A234} ist jeweils ihr Zustandsübergangsdiagramm dargestellt.  
  
 \begin{figure}[h] 
 \centering 
  
 \begin{tikzpicture}[shorten >=1pt,node distance=1.5cm,auto, font=\footnotesize] 
  
 \tikzstyle{every state}=[minimum size = 0cm,inner sep=3pt] 
 \node[state,initial,accepting] (s_0) {$s_0$}; 
 \node[state,accepting] (s_1) [right of = s_0]{$s_1$}; 
 \path[->] (s_0) edge [bend left] node {b} (s_1); 
 \path[->] (s_1) edge [loop right] node {b} (); 
 \path[->] (s_0) edge [loop below] node {a} (); 
 \path[->] (s_1) edge [bend left] node {a} (s_0); 
  
 \node [below left of = s_0] {$A_2$}; 
  
  
 \node[state,initial] (s_0)  [right of = s_0, node distance = 6 cm] {$s_0$}; 
 \node[state,accepting] (s_1) [above right of = s_0]{$s_1$}; 
 \node[state,accepting] (s_2) [below right of = s_0]{$s_2$}; 
 \path[->] (s_0) edge node {b} (s_1); 
 \path[->] (s_1) edge [loop right] node {b} (); 
 \path[->] (s_0) edge node {a} (s_2); 
 \path[->] (s_2) edge [loop right] node {a} (); 
  
 \node [below left of = s_0] {$A_3$}; 
  
  
 \node[state,initial] (s_0) [right of = s_0, node distance = 6 cm]{$s_0$}; 
 \node[state,accepting] (s_1) [above right of = s_0]{$s_1$}; 
 \node[state,accepting] (s_2) [below right of = s_0]{$s_2$}; 
 \path[->] (s_0) edge node {b} (s_1); 
 \path[->] (s_1) edge [loop right] node {b} (); 
 \path[->] (s_0) edge [loop below] node {a} (); 
 \path[->] (s_0) edge node {a} (s_2); 
 \path[->] (s_2) edge [loop right] node {a} (); 
  
 \node [below left of = s_0] {$A_4$}; 
  
 \end{tikzpicture} 
 \caption{Zustandsübergangsdiagramme von $A_2, A_3, A_4$} 
 \label{A234} 
 \end{figure} 
  
  
 \subsection*{Teilaufgabe 5.1.1} 
 [credits = 1 
 ] 
  
 Geben Sie für jeden der Automaten an, ob er ein deterministischer endliche Akzeptor, oder ein nichtdeterministischer endlicher Akzeptor ist.  
  
\textbf{  Lösung : 
}\\
\begin{itemize}
\item $A_1$ ist DEA
\item $A_2$ ist NEA, ohne Papierkorbzustand. Z.b von S1 fehlt ein Übergang mit Beschriftung a, daher brauchen wir Papierkorbzustand, wo ein Übergang mit Beschriftung a hin geht, damit wird dann der Automat DEA
\item $A_3$ ist NEA, von S0 gehen zwei Übergänge mit einer Beschriftung nämlich a

\end{itemize}

  
  
 \subsection*{Teilaufgabe 5.1.2} 
 [credits = 2 
 ] 
  
 Geben Sie für jeden der Automaten  
  
 \begin{enumerate} 
 \item die Zustandsmenge,  
  \item die Finalmenge, 
 \item die Zustandsübergangsfunktion in tabellarischer Form an. 
 \end{enumerate} 
 \textbf{Lösung :}\\
\textit{ Allgemein Zustandmenge : 
 $A = \{S,\Sigma,\sigma,F,start\}$}\\\\
 \begin{itemize}
 \item $A_2 = (\{s_0,s_1\},\{a,b\},\sigma,\{s_0,s_1\},s_0)$ \\
mit $\sigma =$ \\
\begin{tabular}{C|C|C}
S/\Sigma & a & b \\ \hline
s_0 & s_0 & s_1\\
s_1 & s_0 & s_1
\end{tabular}
\item
$A_3 = (\{s_0,s_1,s_2\},\{a,b\},\sigma,\{s_1,s_2\},s_0)$ \\
mit $\sigma =$ \\
\begin{tabular}{C|C|C}
S/\Sigma & a & b \\ \hline
s_0 & s_2 & s_1\\
s_1 & \epsilon & s_1\\
s_2 & s_2 & \epsilon
\end{tabular}
\item
$A_4 = (\{s_0,s_1,s_2\},\{a,b\},\sigma,\{s_1,s_2\},s_0)$ \\
mit $\sigma =$ \\
\begin{tabular}{C|C|C}
S/\Sigma & a & b \\ \hline
s_0 & \{s_0,s_2\}& s_1\\
s_1 & \epsilon & s_1\\
s_2 & s_2 & \epsilon
\end{tabular}

 \end{itemize}


  
  
  
  
  
 \subsection*{Teilaufgabe 5.1.3} 
 [credits = 2, 
 ] 
  
 Geben Sie für jeden der Automaten (wenn möglich) 3 Worte über dem Alphabet $\Sigma = \{a,b\}$ an 
  Benutzen Sie Ihre Ergebnisse, um für jeden Automaten dessen akzeptierte Sprache anzugeben. 
 \begin{enumerate} 
 \item die akzeptiert werden, \\
\textbf{ Lösung :}\\
\begin{itemize}
\item Für Automat $A_2$\\
$w_2 \in \mathcal{L}_2$\\
$w_2 = aaabbbb \ \text{mit} \ \mathcal{L}_2 = \{a^nb^m | n,m \in \mathbb{N}\}$\\
$w_2 = aaababb \ \text{mit} \ \mathcal{L}_2 = \{a^nb^ma^nb^m | n,m \in \mathbb{N}\}$\\
$w_2 = bb  \ \text{mit} \ \mathcal{L}_2 = \{a^nb^m | n \in \mathbb{N}_0,m \in \mathbb{N}\}$
\item Für Automat $A_3$\\
$w_3 \in \mathcal{L}_3$\\
$w_3 = aaa \ \text{mit} \ \mathcal{L}_3 = \{a^n | n \in \mathbb{N}\}$\\
$w_3 = bbbb \ \text{mit} \ \mathcal{L}_3 = \{b^m | m \in \mathbb{N}\}$\\
$w_3 = a  \ \text{mit} \ \mathcal{L}_3 = \{a^n | n = 1 \}$ 
\item Für Automat $A_4$\\
$w_4 \in \mathcal{L}_4$\\
$w_4 = aaa \ \text{mit} \ \mathcal{L}_4 = \{a^n | n \in \mathbb{N}\}$\\
$w_4 = bbbb \ \text{mit} \ \mathcal{L}_4 = \{b^m | m \in \mathbb{N}\}$\\
$w_4 = abb  \ \text{mit} \ \mathcal{L}_4 = \{a^nb^m | n,m \in \mathbb{N} \}$ 
\end{itemize}
 \item die nicht akzeptiert werden. \\
 \textbf{ Lösung :}\\
\begin{itemize}
\item Für Automat $A_2$\\
$w_2 \not \in \mathcal{L}_2$\\
Keine Wörter !
\item Für Automat $A_3$\\
$w_3 \not \in \mathcal{L}_3$\\
$w_3= ba$\\
$w_3=ab$\\
$w_3=aab$ 
\item Für Automat $A_4$\\
$w_4 \not \in \mathcal{L}_4$\\
$w_4= aaba$\\
$w_4=ba$\\
$w_4=aba$ 
\end{itemize}
 \end{enumerate} 
  
  
  
  
  
 \subsection*{Teilaufgabe 5.1.4}[ 
 credits = 2, 
 ] 
  
 Geben Sie für jeden Automaten $\hat{\delta}(s_0,bbb)$ (also den Zustand, in welchem sich der Automat nach Einlesen des Wortes $bbb$ befindet) an. 
  \\
  \textbf{Lösung}:\\
  \begin{itemize}
\item Für Automat $A_2$\\
im Zustand $s_1$
\item Für Automat $A_3$\\
im zustand $s_1$
\item Für Automat $A_4$\\
im Zustand $s_1$
\end{itemize}
  
  
  
 \section*{Aufgabe 5.2}[ 
\textbf{ [topic = Paritätscode 
 ] }
  
 Zur Fehlererkennenung bei einer Datenübertragung wird oft der \emph{Paritätscode} genutzt.  
  
 Die Grundidee des Paritätscodes ist, ausschließlich Datenpakete zu versenden, die eine 
 gerade Anzahl Einsen aufweisen. Hierzu werden die Datenpakete vor dem Versenden um ein 
 Paritätsbit ergänzt, das die Gesamtzahl der Einsen bei Bedarf gerade werden lässt.  
  
 Ein Übertragungsfehler wird dadurch erkannt, dass die Anzahl der Einsen ungerade ist. Ist die Anzahl der Einsen gerade, wurde das Pakte korrekt übertragen. 
  
 Das vor der Übertragung hinzuzufügende Paritätsbit berechnet sich wie folgt:  
  
 \begin{itemize} 
 \item 1, falls die Anzahl der Einsen im Datenpaket ungerade ist 
 \item 0, falls die Anzahl der Einsen im Datenpaket gerade ist 
 \end{itemize} 
  
 Ein Paritätscode der Länge 4 versieht also die ersten 3 Datenbits mit einem 4. Paritätsbit, so dass die Pakete insgesamt wie folgt aussehen:  
  
 \begin{itemize} 
 \item $000|0$ 
 \item $001|1$ 
 \item $010|1$ 
 \item \ldots 
 \end{itemize} 
  
  
 \subsection*{Teiaufgabe 5.2.1}[ 
 credits = 1  
 ] 
  
 Komplettieren Sie die Liste, bis Sie alle 8 verschiedenen Codewörter der Länge 3 um ihr Paritätsbit ergänzt haben. \\ 
  
  \textbf{Lösung}\\
    
 \begin{itemize} 
 \item $000|0$ 
 \item $001|1$ 
 \item $010|1$ 
 \item $011|0$
 \item $100|1$
 \item $101|0$
 \item $110|0$
 \item $111|1$
 \end{itemize}
  
 \subsection*{Teiaufgabe 5.2.2}[ 
 credits = 2  
 ] 
  
 Konstruieren Sie einen DEA $A_P$, der die Integrität eines empfangenen Datenpakets überprüft 
 und alle korrekt übertragenen Wörter akzeptiert. Wurde ein einzelnes Bit des Datenpakets 
 während der Übertragung verfälscht, so soll der Automat das Eingabewort ablehnen. 
  \\ 
\textbf{  Lösung
}\\


  
 \begin{figure}[h] 
   $A_p = (\{s_0,s_1,s_2,s_3,s_4,s_5,s_6,s_7\},\{0,1\},\sigma,\{s_1,s_3,s_5,s_7\},s_0)$ \\ mit $\sigma =$ \\

 \centering 
  
 \begin{tikzpicture}[shorten >=3pt,node distance=2.0cm,auto, font=\footnotesize] 
  
 \tikzstyle{every state}=[minimum size = 0cm,inner sep=3pt] 
 \node[state,initial]   (A) {$s_0$}; 
 \node[state,accepting] (B) [above right of = A]{$s_1$};
 \node[state] 			(C) [ right of = B]{$s_2$};
 \node[state,accepting] (D) [ right of = C]{$s_3$};
  \node[state] 			(E) [below right of = A]{$s_4$};
\node[state,accepting] 	(F) [ right of = E]{$s_5$};
\node[state] 			(G) [ right of = F]{$s_6$};
\node[state,accepting] 	(H) [ right of = G]{$s_7$};

 
 \path[->] (A) edge [bend right] node {0} (B)
  		   	   edge [bend right] node {1} (E) 
  		   (B) edge [bend right] node {1} (C)
  		   	   edge [loop above] node {0} (B) 
  		   (C) edge [bend right] node {1} (D)
  		   	   edge [loop above] node {0} (C) 
  		   (D) edge [loop above] node {0,1} (D)
  		   (E) edge [bend right] node {1} (F)
  		   	   edge [loop below] node {0} (E)
  		   (F) edge [bend right] node {1} (G)
  		   	   edge [loop below] node {0} (F)
  		   (G) edge [bend right] node {1} (H)
  		   	   edge [loop below] node {0} (G)
  		   (H) edge [loop below] node {0,1} (H);

 \node [below left of = A] {$A_p$}; 
   
  
 \end{tikzpicture} 
 \end{figure} 
  
 \subsection*{Teiaufgabe 5.2.3}[ 
 topic = , 
 credits = 1 
 ] 
  
 Wie verhält sich der von Ihnen konstruierte Automat, falls zwei Bits während der Datenübertragung 
 verfälscht wurden? Weist er dieses falsche Wort auch zurück? Begründen Sie Ihre Aussage. \\
  
  \textbf{Lösung}
 \\
 Wenn richtig verstanden hab, also wenn 2 Bits verfälscht werden, könnte eine akzeptierte Bitfolge entstehen, wenn es der Fall ist, dann akzeptiert der Automat die entstandene Bitfolge, sollte die Verfälschung eine nicht akzeptierte Bitfolge entstehen lassen, dann wird der Automat diese ablehnen !\\\\
 \textit{Beispiel} : 0000, Verfälschung von 2 Bits $\rightarrow$ 0101, was es zu Akzeptanz folgt.
 \section*{Aufgabe 5.3}[ 
\textbf{ topic = Die Sprache $L_{11}$ 
 ]} 
  
 Gegeben sei die Sprache $L_{11}$, die alle Wörter über $\{0, 1\}$ enthält, die mit einer 1 beginnen und mit einer 1 
 enden. Formal:  
 \[ 
 L_{11} = \{\omega \in \{0, 1\}^* | \exists \ u \in \{0, 1\}^* : \omega = 1u1 \} 
 \] 
  
 \subsection*{Teiaufgabe 5.3.1} [
 credits = 2 
 ] 
 Geben Sie einen NEA $N_{11}$ an der $L_{11}$ akzeptiert, für den also $\mathcal{L}(N_{11}) = L_{11}$ \\
 \textbf{Lösung} :Sternchen heißt beliebig aber auch NULL?!!!!\\ 
   
 \begin{figure}[h] 
   $N_{11} = (\{s_0,s_1,s_2\},\{0,1\},\sigma,\{s_0,s_2\},s_0)$ \\ mit $\sigma =$ \\

 \centering 
  
 \begin{tikzpicture}[shorten >=3pt,node distance=2.0cm,auto, font=\footnotesize] 
  
 \tikzstyle{every state}=[minimum size = 0cm,inner sep=3pt] 
 \node[state,initial,accepting]   (A) {$s_0$}; 
 \node[state] 		(B) [right of = A]{$s_1$};
 \node[state,accepting] (C) [ right of = B]{$s_2$};

 
 \path[->] (A) edge [bend right] node {1} (B)
  		   	   edge [loop above] node {1} (A) 
  		   (B) edge [bend right] node {1} (C)
  		   	   edge [loop above] node {0} (B) 
  		   (C) edge [loop above] node {1} (C); 
  		  
 \node [below left of = A] {$N_{11}$}; 
   
  
 \end{tikzpicture} 
 \end{figure} 
  
  
  
  
 \subsection*{Teiaufgabe 5.3.2} [
 credits = 2 
 ] 
  
 Geben Sie einen DEA $A_{11}$ an, welcher $L_{11}$ akzeptiert. Tipp: Dieser benötigt genauso viele Zustände wie $N_{11}$, jedoch mehr Zustandsübergänge. \\
 \textbf{Lösung} :   
 \begin{figure}[h] 
   $N_{11} = (\{s_0,s_1,s_2\},\{0,1\},\sigma,\{s_0,s_2\},s_0)$ \\ mit $\sigma =$ \\

 \centering 
  
 \begin{tikzpicture}[shorten >=3pt,node distance=2.0cm,auto, font=\footnotesize] 
  
 \tikzstyle{every state}=[minimum size = 0cm,inner sep=3pt] 
 \node[state,initial,accepting]   (A) {$s_0$}; 
 \node[state] 		(B) [right of = A]{$s_1$};
 \node[state,accepting] (C) [below right  of = B]{$s_2$};

 
 \path[->] (A) edge [bend right] node {0} (B)
  		   	   edge [loop above] node {1} (A) 
  		   (B) edge [bend left] node {1} (C)
  		   	   edge [loop above] node {0} (B) 
  		   (C) edge [loop right] node {1} (C)
  		       edge [bend left] node {0} (B); 

  		  
 \node [below left of = A] {$N_{11}$}; 
   
  
 \end{tikzpicture} 
 \end{figure} 
  
  
  
  
 \subsection*{Teiaufgabe 5.3.3} [
 credits = 2 
 ] 
  
 Geben Sie einen regulären Ausdruck an, der $L_{11}$ erzeugt. Hierbei ist es egal, ob Sie eine formale, oder eine Unix-ähnliche Syntax wählen.  
  
  
  
 \subsection*{Teilaufgabe 5.3.4} [
 credits = 2 
 ] 
  
 Geben Sie eine Grammatik $G_{11}$ an, die $L_{11}$ erzeugt.  \\
  $G_N = \{\{S,A,B,C\},\{0,1\},P,S\}$ mit P =\\
 $S \rightarrow 1A|10B|1C|1$\\
 $A \rightarrow 1S$\\
 $B \rightarrow 10S$\\
 $C \rightarrow 0CS$
 \end{document} 
  
 