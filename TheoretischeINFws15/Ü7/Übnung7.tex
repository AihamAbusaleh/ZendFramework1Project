

\documentclass[12pt]{article}

\usepackage[german]{babel}
\usepackage[utf8]{inputenc}
\usepackage{lmodern}
\usepackage{amsmath}
\usepackage{amssymb}
\usepackage{array}
\usepackage{tikz}
\usepackage{colortbl}
\usepackage{xcolor}
\usetikzlibrary{arrows,automata}
\usepackage{pgf}
\usepackage{amstext}
\def \nat {\mathbf{N}}
\newcolumntype{C}{>{$}c<{$}}
\newcolumntype{R}{>{$}r<{$}}
\usepackage{hyperref}
\usepackage{stmaryrd}
\begin{document}
  
\begin{center}
\begin{huge}
\textbf{Übungsblatt 7}\\
\textit{Automaten aller Art}
\end{huge} \\

Theoretische Informatik\\
Studiengang Angewandte Informatik\\
Wintersemester 2015/2016\\
Prof. Barbara Staehle, HTWG Konstanz
\end{center}

\section*{Aufgabe 7.1}
\textit{[ 
 topic = Ein zellulärer Automat, 
 credits = 4 
 ] }
  
  
 Wir betrachten den zellulären Automaten $A_Z = (Z,S,\nu, \delta)$ mit 
 \begin{itemize} 
 \item $Z = $ eindimensionale Gitterlinie, 
 \item $S = \{0,1\}$, $A$ verwendet also nur die Farben 1 = schwarz, 0 = weiß,  
 \item $\nu$ ist die von-Neumann Nachbarschaft, alle Zellen, die direkt nebeneinander liegen sind benachbart,  
 \item $\delta$ ist gegeben durch die Regeln in Abbildung 1
 \end{itemize} 
  

  
 Geben Sie die 15 Zustände (in schwarz oder weiß eingefärbte Zellen einer Linie mit 31 Feldern an), die der Automat annimmt, nachdem er im durch Abbildung 2 gegebenen Zustand gestartet ist.  
  
 Zeichnen Sie hierfür, beginnend mit Zeile 1, alle Zustände die der Automat mit der Zeit annimmt untereinander. Wenn Sie die Aufgabe mit der Hand lösen, brauchen Sie hierfür ein Karopapier mit $31 \times 16$ Feldern.  
  

  
\section*{Aufgabe 7.2} 
\textit{[
 topic = Schlechte Passwörter 
 ] }
  
 Eine der am häufigsten genutzten Passwörter im deutschsprachigen Raum ist "`qwertz"'. Die IT-Abteilung in der Sie Ihr Praktikum verbringen, möchte deshalb in einem schwach gesicherten System, das als Passwörter nur Kleinbuchstaben erlaubt, alle Passwörter ausmerzen, die gleich, oder ähnlich zu "qwertz"' sind und betreut Sie mit verschiedenen Aufgaben. 
  
  
  
 \subsection*{Teilaufgabe 7.2.1} 
\textit{ [
 topic = Ein DEA für qwertz, 
 credits = 3 
 ] }
  
 Konstruieren Sie einen DEA $A_Q$, der als Eingabe eine beliebig lange Zeichenkette annimmt und diese genau dann akzeptiert, falls Sie identisch zu "`qwertz"' sind. 
  
 Konstruieren Sie Ihren Automaten so, dass er während des Lesens eines Wortes ungleich "`qwertz"' nicht abbricht, sondern das Wort komplett einliest und sich nach Abschluss des Einlesens in einem gesonderten Fehlerzustand befindet. \\
 \textbf{Lösung}:\\
 $M = qwertz$
  \begin{figure}[h] 
 \centering 
  
 \begin{tikzpicture}[shorten >=5pt,node distance=4.5cm,auto, font=\footnotesize] 
  
 \tikzstyle{every state}=[minimum size = 1.0cm,inner sep=5pt] 
 \node[state,initial] (s_0) {$s_0$}; 
  \node[state, accepting] (s_1) [right of = s_0]{$s_1$}; 
  \node[state] (s_F) [below right of = s_0]{$s_F$}; 


  \path[->] (s_0) edge 
   			  node[anchor=above,above]  {M} (s_1)
  		
 (s_0) edge 
   			  node[anchor=below,left]  {$\Sigma \setminus M$} (s_F)

 (s_1) edge 
   			  node[anchor=below,left]  {$\Sigma $} (s_F)
   			  (s_F) edge [loop below] 
   			  node[anchor=below,left]  {$\Sigma $} (s_F);




 \node [below left of = s_0] {$A_{Q}$}; 
   \end{tikzpicture} 
 \caption{Zustandsübergangsdiagramme von $A_{Q}$} 
 \label{A234} 
 \end{figure} 
 
  
  
 \subsection*{Teilaufgabe 7.2.2} 
\textit{[ 
 topic = Ein NEA für *qwertz*, 
 credits = 2 
 ] 
  }
 Konstruieren Sie einen NEA $N_Q$, der als Eingabe eine beliebig lange Zeichenkette annimmt und diese genau dann akzeptiert, falls Sie als Teilwort den String "`qwertz"' enthält. 
  
 Beispiele: qwertz, qwertzz, asdfqwertzuio werden akzeptiert, qwert, quwertz, ertz werden nicht akzeptiert.  \\
  
   \textbf{Lösung}:\\
 $M = qwertz$
  \begin{figure}[h] 
 \centering 
  
 \begin{tikzpicture}[shorten >=5pt,node distance=4.5cm,auto, font=\footnotesize] 
  
 \tikzstyle{every state}=[minimum size = 1.0cm,inner sep=5pt] 
 \node[state,initial] (s_0) {$s_0$}; 
  \node[state, accepting] (s_1) [right of = s_0]{$s_1$}; 


  \path[->] (s_0) edge 
   			  node[anchor=above,above]  {M} (s_1)
  		(s_0) edge [loop below]
   			  node[anchor=below,below]  {$\Sigma$} (s_0)
 

 (s_1) edge [loop below]
   			  node[anchor=below,below]  {$\Sigma $} (s_1);





 \node [below left of = s_0] {$N_{Q}$}; 
   \end{tikzpicture} 
 \caption{Zustandsübergangsdiagramme von $N_{Q}$} 
 \label{A234} 
 \end{figure} 
 
  
  
 \subsection*{Teilaufgabe 7.2.3} 
\textit{[ 
 topic = Ein DEA für *wer*, 
 credits = 3 
 ] }
  
 Konstruieren Sie einen DEA $A_W$, der als Eingabe eine beliebig lange Zeichenkette annimmt und diese genau dann akzeptiert, falls Sie als Teilwort den String "`wer"' enthält. 
  
 Beispiele: wer, qwertz, werwolf werden akzeptiert, we, wetr, erw werden nicht akzeptiert. 
 \newpage
   \textbf{Lösung}:
  \begin{figure}[h] 
 \centering 
  
 \begin{tikzpicture}[shorten >=5pt,node distance=3.5cm,auto, font=\footnotesize] 
  
 \tikzstyle{every state}=[minimum size = 1.0cm,inner sep=3pt] 
 \node[state,initial] (s_0) {$s_0$}; 
  \node[state ] (s_1) [above right of = s_0]{$s_1$}; 
  \node[state ] (s_2) [right of = s_0]{$s_2$}; 
  \node[state, accepting] (s_3) [below right  of = s_0]{$s_3$}; 


  \path[->] (s_0) edge [bend left]
   			  node[anchor=above,above]  {$\epsilon$} (s_1)
  		(s_0) edge [loop below]
   			  node[anchor=below,below]  {$\Sigma \setminus w$} (s_0)
   			  
   			  (s_1) edge [bend left]
   			  node[anchor=above,right]  {e} (s_2)
   			   (s_1) edge 
   			  node[anchor=below,below]  {$\Sigma \setminus e$} (s_0)
  		(s_1) edge [loop above]
   			  node[anchor=above,above]  {w} (s_1)
 
 (s_2) edge 
   			  node[anchor=above,left]  {w} (s_1)
   			  (s_2) edge 
   			  node[anchor=above,left]  {r} (s_3)

 (s_2) edge 
   			  node[anchor=above,above]  {$\Sigma \setminus rw$} (s_0)
 (s_3) edge [loop below]
   			  node[anchor=below,below]  {$\Sigma$} (s_3);

 \node [below left of = s_0] {$A_{w}$}; 
   \end{tikzpicture} 
 \caption{Zustandsübergangsdiagramme von $A_{w}$} 
 \label{A234} 
 \end{figure} 
 
  
   \subsection*{Teilaufgabe 7.2.4} 
\textit{[ 
 topic = Ein DET zur Erkennung von *wer*, 
 credits = 3 
 ] 
  }
 Konstruieren Sie einen DET $T_W$, der als Eingabe eine beliebig lange Zeichenkette annimmt und auf dem Ausgabeband für jedes Zeichen ein Kästen schreibt. $T_W$ soll ein markiertes Kästchen schreiben, wenn als Teilwort der String "`wer"' gelesen wurde. Insbesondere soll jedes Vorkommen von "`wer"' mit einem markierten Kästchen signalisiert werden.Es bleibt Ihnen überlassen, ob Sie $T_W$ als Mealy- oder als Moore-Automat konstruieren. 
  
 Beispiele:  
 \begin{itemize} 
 \item wer $\rightarrow \Box\Box\boxtimes$ 
 \item qwertz $\rightarrow \Box\Box\Box\boxtimes\Box\Box$ 
 \item werwolfwer $\rightarrow \Box\Box\boxtimes\Box\Box\Box\Box\Box\Box\boxtimes$ 
 \item wetr $\rightarrow \Box\Box\Box\Box$ 
 \end{itemize} 

  
   
  
 \end{document} 
  
 