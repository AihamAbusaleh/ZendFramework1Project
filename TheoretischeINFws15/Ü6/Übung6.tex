

\documentclass[12pt]{article}

\usepackage[german]{babel}
\usepackage[utf8]{inputenc}
\usepackage{lmodern}
\usepackage{amsmath}
\usepackage{amssymb}
\usepackage{array}
\usepackage{tikz}
\usepackage{colortbl}
\usepackage{xcolor}
\usetikzlibrary{arrows,automata}
\usepackage{pgf}
\usepackage{amstext}
\def \nat {\mathbf{N}}
\newcolumntype{C}{>{$}c<{$}}
\newcolumntype{R}{>{$}r<{$}}
\usepackage{hyperref}
\usepackage{stmaryrd}
\begin{document}
  
\begin{center}
\begin{huge}
\textbf{Übungsblatt 6}\\
\textit{Kellerautomaten}
\end{huge} \\

Theoretische Informatik\\
Studiengang Angewandte Informatik\\
Wintersemester 2015/2016\\
Prof. Barbara Staehle, HTWG Konstanz
\end{center}

\section*{Aufgabe 6.1}

\textit{[ 
 topic = Arbeitsweise des PDAs $P_1$, 
 credits = 3 
 ] }
  \\
 Erinnern Sie sich an den in der Vorlesung definierten PDA \\ $P_1 = (\{s_0, s_1\}, \{a,b\}, \{A, \bot\}, \delta, s_0)$ mit $\mathcal{L}(P_1) = \{a^nb^n \mid n \in \nat \}$. 
  
 Bestimmen Sie für die Worte  
  
 \begin{enumerate} 
 \item $\omega_1 = aabbb$ 
\begin{large}
\textbf{ $\lightning$}\\
\end{large}

\textbf{Lösung}\\
\begin{tabular}{|C||C|C|C|}
\centering
Schritt & Zustand & w & Keller\\ \hline
0 & s_0 & aabbb & \bot\\
1& s_0 & abbb & A\bot\\
2& s_0 & bbb & AA\bot\\
3& s_1 & bb & A\bot\\
4& s_1 & b & \bot\\
\rowcolor[gray]{.7}
5& s_1 & b & \epsilon\\
\end{tabular}\\\\

das letzte b (markierte Zeile) kann nicht gelesen werden, da der Zustand $b,\bot,\epsilon$ \emph{NICHT} existiert ...
Also anders gesagt, b kann erst dann gelesen werden wenn im Keller A steht und nicht $\bot$.. 
 \item $\omega_2 = aaabbb$ $\surd$
 
 
\textbf{Lösung}\\
\begin{tabular}{|C||C|C|C|}
\centering
Schritt & Zustand & w & Keller\\ \hline
0 & s_0 & aaabbb & \bot\\
1& s_0 &  aabbb & A\bot\\
2& s_0 &  abbb & AA\\
3& s_0 &   bbb & AAA\\
4& s_1 &   bb & AA\\
5& s_1 & b & A\\
6& s_1 & \epsilon & \bot\\
7& s_1 & \epsilon & \epsilon
\end{tabular}\\\\
Keller ist leer, das Wort w wurde akzeptiert..
 \item $\omega_3 = abaabb$ \begin{large}
\textbf{ $\lightning$}\\
\end{large}
 
\textbf{Lösung}\\
\begin{tabular}{|C||C|C|C|}
\centering
Schritt & Zustand & w & Keller\\ \hline
0 & s_0 & abaabb & \bot\\
1& s_0 &   baabb & A\bot\\
2& s_1 &    aabb & \bot\\
\rowcolor[gray]{.7}
3& s_1 & aabb & \bot
\end{tabular}\\\\
Von Zustand $s_1$ (markiert )kann kein a gelesen werden daher w wird nicht akzeptiert 
 \end{enumerate} 
  
 jeweils alle Konfigurationen (aktueller Zustand, verbleibendes Eingabewort, Inhalt des Kellers), die $P_1$ während der Verarbeitung der Worte durchläuft. Beantworten Sie anschließend, warum die Worte (nicht) akzeptiert werden. 
  
 Hinweis: Verwenden Sie Tabellen, oder die Übergangsrelation. 
  
  \section*{Aufgabe 6.2}

\textit{  
[ 
 topic = Ein PDA für die OTTO-Zahlen 
 ] }\\
  
 Erinnern Sie sich an die OTTO-Zahlen vom 3. Übungsblatt. Wir betrachten jetzt allerdings nur OTTO-Zahlen mit dem Ziffernvorrat 1-3: $L_{O3} \subseteq \{1,2,3\}^*$ mit \\ $L_{O3} = \{1, 2, 3, 11, 22, 33, 111, 121, 131 \ldots 2332 \ldots 132321, \ldots \}$.  
  
 \subsection*{Teilaufgabe 6.2.1} 
 
  
 Geben Sie den PDA $P_{O3}$ an, der $L_{O3}$ akzeptiert. Lassen Sie sich hierfür von den PDAs, die Sie in der Vorlesung kennengelernt haben, inspirieren.  
  
 Geben Sie die Zustandsübergangsfunktion sowohl in tabellarischer bzw. formaler Form als auch mittels eines erweiterten Zustandsübergangsdiagramms an. \\
  \textbf{Lösung:}\\
 \\
 $P_{otto} = (\{s_0, s_1\}, \{1,2,3\}, \{X\}, \delta, s_0)$\\ \text{mit} \ $X \in \{A,B,C, \bot\}$\\\\
 $\delta(s_0,1,X):=\{s_0,AX\}
\\ 
 \delta(s_0,2,X):=\{s_0,BX\}
\\ 
 \delta(s_0,3,X):=\{s_0,CX\}
 \\
 \delta(s_0,X,X):=\{s_1,X\}
 \\
 \delta(s_1,1,A):=\{s_1,\epsilon\}
 \\
 \delta(s_1,2,B):=\{s_1,\epsilon\}
 \\
 \delta(s_1,3,C):=\{s_1,\epsilon\}
 \\
  \delta(s_1,\epsilon,\bot):=\{s_1,\epsilon\}$
\\
 \begin{figure}[h] 
 \centering 
  
 \begin{tikzpicture}[shorten >=7pt,node distance=4.5cm,auto, font=\footnotesize] 
  
 \tikzstyle{every state}=[minimum size = 1cm,inner sep=10pt] 
 \node[state,initial] (s_0) {$s_0$}; 
  \node[state] (s_1) [right of = s_0]{$s_1$}; 
 %  \node[state,yshift=+2cm] (s_1) [above of = s_0]{$s_1$}; 


 %\path[->] (s_0) edge [bend right=-70] node[anchor=below,below] {$1|2|3,				A;AA$} node {$1|2|3,\bot;A$} (s_1); 
 %\path[->] (s_1) edge [loop above] node[anchor=north,right]  {$1|2|3,		A;\epsilon$} (); 
  \path[->] (s_0) edge [loop below] node[anchor=left,left]  {$1,X;AX$} node[anchor=right,right]     {$2,X;BX$} ();
  
   \path[->] (s_0) edge [loop above] node[anchor=above,above]  {$3,X;CX$}  (); 
 
  %\path[->] (s_1) edge  node {$1|2|3,				A;A$} (s_0);
 \path[->] (s_1) edge [loop below] node[anchor=left,left]  {$1,A;\epsilon$} node[anchor=right,right]    {$2,B;\epsilon$}();
 
  \path[->] (s_1) edge [loop above] node[anchor=above,above]  {$3,C;\epsilon$}
 ();
   \path[->] (s_1) edge [loop right] node[anchor=right,right]  {$\epsilon,\bot;\epsilon$}  
 ();

  
  \path[->] (s_0) edge [bend left] node[anchor=above,above] {$\epsilon,X;X$} (s_1);


 \node [below left of = s_0] {$P_{otto}$}; 
   \end{tikzpicture} 
 \caption{Zustandsübergangsdiagramme von $P_{otto}$} 
 \label{A234} 
 \end{figure} 
  
 \subsection*{Teilaufgabe 6.2.2} 

 Bestimmen Sie für die Worte  
  
 \begin{enumerate} 
 \item $\omega_1 = 123321$ $\surd$ \\
 
\textbf{Lösung}\\
\begin{tabular}{|C||C|R|R|}
\centering
Schritt & Zustand & w & Keller\\ \hline
0 & s_0 & 123321 & \bot\\
1& s_0 &   23321 & A \bot\\
2& s_0 &    3321 &  BA\bot\\
3& s_0 &     321 &  CBA\bot\\
4& s_1 &      21 &  BA\bot\\
5 & s_1 &      1 &  A \bot\\
6 & s_1 &     \epsilon & \bot\\
7 & s_1 & \epsilon & \epsilon
\end{tabular}\\\\
Keller ist leer, das Wort w wurde akzeptiert..
 \item $\omega_2 = 321311$ $\lightning$ \\
 \textbf{Lösung}\\
 keine OTTO Zahl 
 \end{enumerate} 
  
 jeweils alle Konfigurationen, die $P_1$ während der Verarbeitung der Worte \textbf{auf einem möglichen Pfad} durchläuft. Falls es einen akzeptierenden Pfad gibt, so wählen Sie bitte diesen. Beantworten Sie anschließend, warum die Worte (nicht) akzeptiert werden. 
  
  
  
 \end{document} 
  